\documentclass[uplatex]{jsarticle}
\usepackage{url}

\begin{document}
Rでベイス線形回帰

PRML 3.3.2章 予測分布
Rでの実装について事例紹介をします。
参考にしたのは、\url{http://d.hatena.ne.jp/n_shuyo/20090709/predictive}です。

予測分布の式。
\begin{equation}
p(t|\mathbf{x},\mathbf{t},α,β)=\mathit{N}(t|\mathbf{m}^{T}_{N}φ(\mathbf{x}),σ^{2}_{N}(\mathbf{x}))
\end{equation}

念のためガウス分布の定義。教科書では(1.47)。
$μ$は平均、$σ^2$が分散。
\begin{equation}
N(x|μ,σ^2)=\frac{1}{(2πσ^2)^{1/2}}exp\biggl\{-\frac{1}{2σ^2}(x-μ)^2\biggl\}
\end{equation}

ここで、
\begin{equation}
σ^2_N(\mathbf{x})=\frac{1}{β}+φ(\mathbf{x})^T\mathbf{S}_Nφ(x)
\end{equation}

\begin{equation}
\mathbf{m}_N=β\mathbf{S}_N\mathbf{φ}^T\mathbf{t}
\end{equation}

\begin{equation}
\mathbf{S}^{-1}_N=α\mathbf{I}+β\mathbf{φ}^T\mathbf{φ}
\end{equation}

基底関数はガウス基底関数を使用する。教科書では(3.4)。
\begin{equation}
φ_j(x)=exp\biggl\{-\frac{(x-u_j)^2}{2s^2}\biggl\}
\end{equation}
ここで$μ_j$は入力空間における基底関数の位置を表し、パラメータsは空間的な尺度を表す。


\end{document}
