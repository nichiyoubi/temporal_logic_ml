\documentclass[uplatex]{jsarticle}
\usepackage{url}
\usepackage{listings}

\title{Rでベイス線形回帰}
\author{宇佐美雅紀}
\date{2014/12/01}

\begin{document}
\maketitle

\section{はじめに}
この資料は、第8回 機械学習勉強会で勉強した、ベイス線形回帰をR言語を使って実装する例を説明するものです。参考にしたのは、以下のブログです。\par

\begin{center}
{\large Mimanca qualche giovedi?}\par
Rでベイズ線形回帰の予測分布(2009-07-09付)
\url{http://d.hatena.ne.jp/n_shuyo/20090709/predictive}
\end{center}
\par
ソースコードは、丸ごとコピペです。宇佐美のオリジナリティはありません。\par
コンセプトは、{\Large 「他人のふんどしで相撲をとる」}です。


\section{PRML 3.3.2章 予測分布のおさらい}
教科書の3.3.2章で解説している予測分布の式をおさらいします。予測分布の式は以下のようになります(教科書 3.58式)。
\begin{equation}
p(t|\mathbf{x},\mathbf{t},α,β)=\mathit{N}(t|\mathbf{m}^{T}_{N}φ(\mathbf{x}),σ^{2}_{N}(\mathbf{x}))
\end{equation}
\\
ここで、予測分布の分散$σ^2_N(x)$は、以下の式で与えられます(教科書 3.59式)。
\begin{equation}
σ^2_N(\mathbf{x})=\frac{1}{β}+φ(\mathbf{x})^T\mathbf{S}_Nφ(x)
\end{equation}
\\
また、$\mathbf{m}_N$、$\mathbf{S}^{-1}_N$はそれぞれ教科書3.53式、3.54式から以下のように与えられます。
\begin{equation}
\mathbf{m}_N=β\mathbf{S}_N\mathbf{φ}^T\mathbf{t}
\end{equation}
\begin{equation}
\mathbf{S}^{-1}_N=α\mathbf{I}+β\mathbf{φ}^T\mathbf{φ}
\end{equation}
\\
基底関数はガウス基底関数を使用します(教科書 3.4式)。
ここで$μ_j$は入力空間における基底関数の位置を表し、パラメータsは空間的な尺度を表す、と教科書には書いてますがイマイチよくわかりません。
\begin{equation}
φ_j(x)=exp\biggl\{-\frac{(x-u_j)^2}{2s^2}\biggl\}
\end{equation}
\\
念のためガウス分布の定義もおさらいします(教科書 1.47式)。
$μ$は平均、$σ^2$が分散。
\begin{equation}
N(x|μ,σ^2)=\frac{1}{(2πσ^2)^{1/2}}exp\biggl\{-\frac{1}{2σ^2}(x-μ)^2\biggl\}
\end{equation}


\section{Rのソースコード}
\begin{lstlisting}[basicstyle=\ttfamily\footnotesize, frame=single]
#####################################################
#
# Usage: R --vanilla --slave < linear_reqgression.r
# 	http://d.hatena.ne.jp/n_shuyo/20090709/predictive
#
#####################################################

M <- 9		# number of basis function
alpha <- 2	# hyper parameter α
beta <- 25	# hyper parameter β
Lattice <- 30	# number of graph's lattice
s<-0.1
u_i<-0.5

# training data
xlist <- seq(0, 1, length=25)
tlist <- sin(2*pi*xlist) + rnorm(length(xlist), sd=0.2)
D0 <- data.frame(x=xlist, t=tlist)

predictive <- function(D) {
    # design matrix
    phi <- function(x) sapply(x,function(x){exp(-(x-seq(0,1,length=9))^2/(2*s*s))})
    PHI <- t(phi(D$x))

    # convariance matrix & means
    S_N_inv <- alpha * diag(9) + beta * t(PHI) %*% PHI
    S_N <- solve(S_N_inv)
    m_N <- beta * S_N %*% t(PHI) %*% D$t

    # regression function
    y <- function(x)(t(phi(x)) %*% m_N)
    plot(y, xlim=c(0,1), ylim=c(-1.2, 1.2))
    par(new=T)
    plot(D, xlim=c(0,1), ylim=c(-1.2, 1.2), ylab="")

    # predictive distribution
    var_N <- function(x) {1/beta + (t(phi(x)) %*% S_N %*% phi(x))[1]}
    function(x,t) {
        mapply(function(x,t)dnorm(t,m=(t(m_N) %*% phi(x))[1], s=var_N(x), log=T), x, t)
    }
}
draw_dist <- function(p){
    x <- seq(0, 1, length=Lattice)
    t <- seq(-1.5, 1.5, length=Lattice*2)
    z <- outer(x, t, p)
    persp(x, t, z, theta=0, phi=60, shade=0.4)
}

p <- predictive(D0[sample(length(D0$x))[1:1],])
draw_dist(p);
p <- predictive(D0[sample(length(D0$x))[1:2],])
draw_dist(p);
p <- predictive(D0[sample(length(D0$x))[1:4],])
draw_dist(p);
p <- predictive(D0)
draw_dist(p);
\end{lstlisting}

\section{解説}
\subsection{計画行列}
ガウス基底関数をおさらいします。
\begin{equation}
φ_j(x)=exp\biggl\{-\frac{(x-u_j)^2}{2s^2}\biggl\}
\end{equation}

計画行列のおさらいです。
\begin{equation}
\mathbf{φ}=\left(
\begin{array}{cccc}
φ_0(\mathbf{x}_1) & φ_1(\mathbf{x}_1) & … & φ_{M-1}(\mathbf{x}_1)\\
φ_0(\mathbf{x}_2) & φ_1(\mathbf{x}_2) & … & φ_{M-1}(\mathbf{x}_2)\\
… & … & … & …\\
φ_0(\mathbf{x}_N) & φ_1(\mathbf{x}_N) & … & φ_{M-1}(\mathbf{x}_N) 
\end{array}
\right)
\end{equation}

\begin{lstlisting}[basicstyle=\ttfamily\footnotesize, frame=single]
# design matrix
phi <- function(x) sapply(x,function(x){exp(-(x-seq(0,1,length=9))^2/(2*s*s))})
PHI <- t(phi(D$x))
\end{lstlisting}

計画行列に相当する部分のソースコードです。\par
phiの後半部分は、ガウス基底関数そのままであることがわかります。ガウス基底関数の$μ_j$部分がseq(0,1,length=9)に変更されています。これは、パラメータ$μ_j$を[0,1]で9等分にすることを意味しています。これでガウス基底関数部分はスカラー値xを与えると、ベクトル値が返る関数になります。sapplyは、リストに関数を適用して行列を返す関数なので、phiにリストを与えると行列が返ります。\par
phi(D\$x)で、データフレームDのx列をphi関数に適用して、計画行列を作ります。
t関数は行列を転置してくれる関数です。

\subsection{共分散\&平均行列}
共分散$S_N$と平均行列$M_N$をおさらいします。
\begin{equation}
\mathbf{m}_N=β\mathbf{S}_N\mathbf{φ}^T\mathbf{t}
\end{equation}
\begin{equation}
\mathbf{S}^{-1}_N=α\mathbf{I}+β\mathbf{φ}^T\mathbf{φ}
\end{equation}
\begin{lstlisting}[basicstyle=\ttfamily\footnotesize, frame=single]
    # convariance matrix & means
    S_N_inv <- alpha * diag(9) + beta * t(PHI) %*% PHI
    S_N <- solve(S_N_inv)
    m_N <- beta * S_N %*% t(PHI) %*% D$t
\end{lstlisting}

共分散$S_N$と平均行列$M_N$に相当する部分のソースコードです。\par
diagは単位行列を作る関数です。solveは逆行列を求める関数です。\par
ハイパーパラメータのαとβは、ソースコードの最初で定数として定義されています。

\subsection{予測分布}
予測分布の式をおさらいします。
\begin{equation}
p(t|\mathbf{x},\mathbf{t},α,β)=\mathit{N}(t|\mathbf{m}^{T}_{N}φ(\mathbf{x}),σ^{2}_{N}(\mathbf{x}))
\end{equation}
予測分布の分散$σ^2_N(x)$の式をおさらいします。
\begin{equation}
σ^2_N(\mathbf{x})=\frac{1}{β}+φ(\mathbf{x})^T\mathbf{S}_Nφ(x)
\end{equation}
\begin{lstlisting}[basicstyle=\ttfamily\footnotesize, frame=single]
    # predictive distribution
    var_N <- function(x) {1/beta + (t(phi(x)) %*% S_N %*% phi(x))[1]}
    function(x,t) {
        mapply(function(x,t)dnorm(t,m=(t(m_N) %*% phi(x))[1], s=var_N(x), log=T), x, t)
    }
\end{lstlisting}

予測分布の式に相当する部分のソースコードです。\par
分散の関数は、式そのままです。\par
dnorm関数は、正規分布の確率密度です。

\end{document}
